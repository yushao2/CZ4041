\documentclass{article}

\usepackage{fancyhdr}
\usepackage{extramarks}
\usepackage{amsmath}
\usepackage{amsthm}
\usepackage{amsfonts}
\usepackage{tikz}
\usepackage[plain]{algorithm}
\usepackage{algpseudocode}
\usepackage{enumerate}

\usetikzlibrary{automata,positioning}

%
% Basic Document Settings
%

\topmargin=-0.45in
\evensidemargin=0in
\oddsidemargin=0in
\textwidth=6.5in
\textheight=9.0in
\headsep=0.25in

\linespread{1.1}

\pagestyle{fancy}
\lhead{\hmwkAuthorName}
\chead{\hmwkClass\ (\hmwkClassInstructor\ \hmwkClassTime): \hmwkTitle}
\rhead{\firstxmark}
\lfoot{\lastxmark}
\cfoot{\thepage}

\renewcommand\headrulewidth{0.4pt}
\renewcommand\footrulewidth{0.4pt}

\setlength\parindent{0pt}

%
% Create Problem Sections
%

\newcommand{\enterProblemHeader}[1]{
    \nobreak\extramarks{}{Problem \arabic{#1} continued on next page\ldots}\nobreak{}
    \nobreak\extramarks{Problem \arabic{#1} (continued)}{Problem \arabic{#1} continued on next page\ldots}\nobreak{}
}

\newcommand{\exitProblemHeader}[1]{
    \nobreak\extramarks{Problem \arabic{#1} (continued)}{Problem \arabic{#1} continued on next page\ldots}\nobreak{}
    \stepcounter{#1}
    \nobreak\extramarks{Problem \arabic{#1}}{}\nobreak{}
}

\setcounter{secnumdepth}{0}
\newcounter{partCounter}
\newcounter{homeworkProblemCounter}
\setcounter{homeworkProblemCounter}{1}
\nobreak\extramarks{Problem \arabic{homeworkProblemCounter}}{}\nobreak{}

%
% Homework Problem Environment
%
% This environment takes an optional argument. When given, it will adjust the
% problem counter. This is useful for when the problems given for your
% assignment aren't sequential. See the last 3 problems of this template for an
% example.
%
\newenvironment{homeworkProblem}[1][-1]{
    \ifnum#1>0
        \setcounter{homeworkProblemCounter}{#1}
    \fi
    \section{Problem \arabic{homeworkProblemCounter}}
    \setcounter{partCounter}{1}
    \enterProblemHeader{homeworkProblemCounter}
}{
    \exitProblemHeader{homeworkProblemCounter}
}

%
% Homework Details
%   - Title
%   - Due date
%   - Class
%   - Section/Time
%   - Instructor
%   - Author
%

\newcommand{\hmwkTitle}{Tutorial Week 2}
\newcommand{\hmwkDueDate}{January 21, 2021}
\newcommand{\hmwkClass}{CZ4041}
\newcommand{\hmwkClassTime}{CS4}
\newcommand{\hmwkClassInstructor}{Assoc Prof Pan, Sinno Jialin}
\newcommand{\hmwkAuthorName}{\textbf{Pang Yu Shao}}
\newcommand{\hmwkAuthorID}{\textbf{U1721680D}}

%
% Title Page
%

\title{
    \vspace{2in}
    \textmd{\textbf{\hmwkClass:\ \hmwkTitle}}\\
    \normalsize\vspace{0.1in}\small{Due\ on\ \hmwkDueDate\ at 8:30am}\\
    \vspace{0.1in}\large{\textit{\hmwkClassInstructor\ - \hmwkClassTime}}
    \vspace{3in}\\
    \hmwkAuthorName\\
    \hmwkAuthorID
}

\date{21/01/2021}

\renewcommand{\part}[1]{\textbf{\large Part \Alph{partCounter}}\stepcounter{partCounter}\\}

%
% Various Helper Commands
%

% Useful for algorithms
\newcommand{\alg}[1]{\textsc{\bfseries \footnotesize #1}}

% For derivatives
\newcommand{\deriv}[1]{\frac{\mathrm{d}}{\mathrm{d}x} (#1)}

% For partial derivatives
\newcommand{\pderiv}[2]{\frac{\partial}{\partial #1} (#2)}

% Integral dx
\newcommand{\dx}{\mathrm{d}x}

% Alias for the Solution section header
\newcommand{\solution}{\textbf{\large Solution}}

% Probability commands: Expectation, Variance, Covariance, Bias
\newcommand{\E}{\mathrm{E}}
\newcommand{\Var}{\mathrm{Var}}
\newcommand{\Cov}{\mathrm{Cov}}
\newcommand{\Bias}{\mathrm{Bias}}

\begin{document}

\maketitle

\pagebreak

\begin{homeworkProblem}
    Show how to estimate \(P(Y=0|X=0)\) and \(P(Y=2|X=0)\) on the 17th 
    page of the lecture notes "Lecture 2a".\\\\

    \textbf{Solution}\\
    \textbf{For} \(\mathbf{P(Y=0|X=0)}\):\\
    \[
        \begin{split}
        P(Y=0|X=0) &= \frac{P(X=0|Y=0) \times P(Y=0)}{P(X=0)}
        \\\\
        &= \frac{P(X=0|Y=0) \times P(Y=0)}{P(X=0, Y=0) + P(X=0, Y=1) + P(X=0, Y=2)}
        \\\\
        &= \frac{P(X=0|Y=0) \times P(Y=0)}{P(X=0|Y=0) \times P(Y=0) + 
                                           P(X=0|Y=1) \times P(Y=1) + 
                                           P(X=0|Y=2) \times P(Y=2)}
        \\\\
        &= \frac{0.54 \times 0.39}{0.54 \times 0.39 + 
                                           0.44 \times 0.3 + 
                                           0.49 \times 0.31}
        \\\\
        &= \frac{0.2106}{0.4945}
        \\
        &=\mathbf{0.426}
        \end{split}
    \]
    \\\\
    \textbf{For} \(\mathbf{P(Y=2|X=0)}\):\\
    \[
        \begin{split}
        P(Y=2|X=0) &= \frac{P(X=0|Y=2) \times P(Y=2)}{P(X=0)}
        \\\\
        &= \frac{P(X=0|Y=2) \times P(Y=2)}{P(X=0, Y=0) + P(X=0, Y=1) + P(X=0, Y=2)}
        \\\\
        &= \frac{0.49 \times 0.31}{0.4945}
        \\\\
        &= \frac{0.1519}{0.4945}
        \\
        &=\mathbf{0.307}
        \end{split}
    \]



\end{homeworkProblem}

\pagebreak
\begin{homeworkProblem}
    Suppose that if a person has lung cancer, his/her probability of having
    gene X is 0.9, and if a person does not have lung cancer, his/her probability of having
    gene X is 0.2. The probability of a person having lung cancer is 0.01. Now, we know
    that a patient A has gene X
    \begin{enumerate}
        \item Please use Bayesian decision theory with 0/1 loss to predict whether the patient
        A has lung cancer or notes
        \item Consider that costs of misclassification are different. Assume that the cost for
        correct decisions is 0, the cost of misclassifying a person who does not have lung
        cancer to be a person with lung cancer is 0.007, and the cost of misclassifying
        a person who has lung cancer to be a healthy patient is 1. Please use Bayesian
        decision theory with the predefined loss to predict whether the patient A has lung
        cancer or not.
    \end{enumerate}

    \textbf{Solution}\\
    \textbf{Part 1}\\
    First, list down all information:
    \begin{itemize}
        \item \(P(X=1|LC=1) = 0.9\)
        \item \(P(X=1|LC=0) = 0.2\)
        \item \(P(LC=1) = 0.01\)
        \item \(P(LC=0) = 0.99\)
    \end{itemize}
    Using Bayes Rule:
    \begin{itemize}
        \item \(P(LC=0|X=1) = \frac{P(X=1|LC=0) \times P(LC=0)}{P(X=1)} = 
              \frac{0.198}{P(X=1)}\)
        \item \(P(LC=1|X=1) = \frac{P(X=1|LC=1) \times P(LC=1)}{P(X=1)} = 
              \frac{0.009}{P(X=1)}\)
    \end{itemize}
    Risk for taking action \(a_i\):
    \(R(a_i|x) = \displaystyle\sum_{c=0}^{C-1} \lambda_{ic} P(y=c|x)\)
    \\\\
    Risk of predicting LC = 1 given X = 1:
    \[
        \begin{split}
        R(a_{LC=1}|X=1) &= \displaystyle\sum_{c=0}^{C-1} \lambda_{ic} P(LC=c|X=1)
        \\
        &= 1*(P(LC=0|X=1)) + 0*(P(LC=1|X=1))
        \\
        &= 0.198/P(X=1)
        \end{split}
    \]
    Risk of predicting LC = 0 given X = 1:
    \[
        \begin{split}
        R(a_{LC=0}|X=1) &= \displaystyle\sum_{c=0}^{C-1} \lambda_{ic} P(LC=c|X=1)
        \\
        &= 0*(P(LC=0|X=1)) + 1*(P(LC=1|X=1))
        \\
        &= \mathbf{0.009}/P(X=1)
        \end{split}
    \]

    Patient A is predicted to not have Lung Cancer

    \pagebreak
    \textbf{Part 2}\\
    Risk of predicting LC = 1 given X = 1 (misclassification cost = 0.007):
    \[
        \begin{split}
        R(a_{LC=1}|X=1) &= \displaystyle\sum_{c=0}^{C-1} \lambda_{ic} P(LC=c|X=1)
        \\
        &= 0.007*(P(LC=0|X=1)) + 0*(P(LC=1|X=1))
        \\
        &= \mathbf{0.001386}/P(X=1)
        \end{split}
    \]
    Risk of predicting LC = 0 given X = 1 (misclassification cost = 1):
    \[
        \begin{split}
        R(a_{LC=0}|X=1) &= \displaystyle\sum_{c=0}^{C-1} \lambda_{ic} P(LC=c|X=1)
        \\
        &= 0*(P(LC=0|X=1)) + 1*(P(LC=1|X=1))
        \\
        &= 0.009/P(X=1)
        \end{split}
    \]


    Patient A is predicted to have Lung Cancer

\end{homeworkProblem}

\pagebreak


    



\end{document}