\documentclass{article}

\usepackage{fancyhdr}
\usepackage{extramarks}
\usepackage{amsmath}
\usepackage{amsthm}
\usepackage{amsfonts}
\usepackage{tikz}
\usepackage[plain]{algorithm}
\usepackage{algpseudocode}
\usepackage{enumerate}

\usetikzlibrary{automata,positioning}

%
% Basic Document Settings
%

\topmargin=-0.45in
\evensidemargin=0in
\oddsidemargin=0in
\textwidth=6.5in
\textheight=9.0in
\headsep=0.25in

\linespread{1.1}

\pagestyle{fancy}
\lhead{\hmwkAuthorName}
\chead{\hmwkClass\ (\hmwkClassInstructor\ \hmwkClassTime): \hmwkTitle}
\rhead{\firstxmark}
\lfoot{\lastxmark}
\cfoot{\thepage}

\renewcommand\headrulewidth{0.4pt}
\renewcommand\footrulewidth{0.4pt}

\setlength\parindent{0pt}

%
% Create Problem Sections
%

\newcommand{\enterProblemHeader}[1]{
    \nobreak\extramarks{}{Problem \arabic{#1} continued on next page\ldots}\nobreak{}
    \nobreak\extramarks{Problem \arabic{#1} (continued)}{Problem \arabic{#1} continued on next page\ldots}\nobreak{}
}

\newcommand{\exitProblemHeader}[1]{
    \nobreak\extramarks{Problem \arabic{#1} (continued)}{Problem \arabic{#1} continued on next page\ldots}\nobreak{}
    \stepcounter{#1}
    \nobreak\extramarks{Problem \arabic{#1}}{}\nobreak{}
}

\setcounter{secnumdepth}{0}
\newcounter{partCounter}
\newcounter{homeworkProblemCounter}
\setcounter{homeworkProblemCounter}{1}
\nobreak\extramarks{Problem \arabic{homeworkProblemCounter}}{}\nobreak{}

%
% Homework Problem Environment
%
% This environment takes an optional argument. When given, it will adjust the
% problem counter. This is useful for when the problems given for your
% assignment aren't sequential. See the last 3 problems of this template for an
% example.
%
\newenvironment{homeworkProblem}[1][-1]{
    \ifnum#1>0
        \setcounter{homeworkProblemCounter}{#1}
    \fi
    \section{Problem \arabic{homeworkProblemCounter}}
    \setcounter{partCounter}{1}
    \enterProblemHeader{homeworkProblemCounter}
}{
    \exitProblemHeader{homeworkProblemCounter}
}

%
% Homework Details
%   - Title
%   - Due date
%   - Class
%   - Section/Time
%   - Instructor
%   - Author
%

\newcommand{\hmwkTitle}{Tutorial Week 4}
\newcommand{\hmwkDueDate}{February 4, 2021}
\newcommand{\hmwkClass}{CZ4041}
\newcommand{\hmwkClassTime}{CS4}
\newcommand{\hmwkClassInstructor}{Assoc Prof Pan, Sinno Jialin}
\newcommand{\hmwkAuthorName}{\textbf{Pang Yu Shao}}
\newcommand{\hmwkAuthorID}{\textbf{U1721680D}}

%
% Title Page
%

\title{
    \vspace{2in}
    \textmd{\textbf{\hmwkClass:\ \hmwkTitle}}\\
    \normalsize\vspace{0.1in}\small{Due\ on\ \hmwkDueDate\ at 8:30am}\\
    \vspace{0.1in}\large{\textit{\hmwkClassInstructor\ - \hmwkClassTime}}
    \vspace{3in}\\
    \hmwkAuthorName\\
    \hmwkAuthorID
}

\date{04/02/2021}

\renewcommand{\part}[1]{\textbf{\large Part \Alph{partCounter}}\stepcounter{partCounter}\\}

%
% Various Helper Commands
%

% Useful for algorithms
\newcommand{\alg}[1]{\textsc{\bfseries \footnotesize #1}}

% For derivatives
\newcommand{\deriv}[1]{\frac{\mathrm{d}}{\mathrm{d}x} (#1)}

% For partial derivatives
\newcommand{\pderiv}[2]{\frac{\partial}{\partial #1} (#2)}

% Integral dx
\newcommand{\dx}{\mathrm{d}x}

% Alias for the Solution section header
\newcommand{\solution}{\textbf{\large Solution}}

% Probability commands: Expectation, Variance, Covariance, Bias
\newcommand{\E}{\mathrm{E}}
\newcommand{\Var}{\mathrm{Var}}
\newcommand{\Cov}{\mathrm{Cov}}
\newcommand{\Bias}{\mathrm{Bias}}

\begin{document}

\maketitle

\pagebreak

\begin{homeworkProblem}
    \begin{enumerate}
        \item Estimate the conditionaal probabilities for \(P(A=1|+)\), \(P(B=1|+)\), \(P(C=1|+)\), \(P(A=1|-)\),
        \(P(B=1|-)\), \(P(C=1|-)\)
        \item Use the estimate of conditional probabilities given in the previous question to
        predict the class label for a test example \((A = 1, B = 1, C = 1)\) using the naive
        Bayes approach.        
    \end{enumerate}
    

    \textbf{Solution}\\
    1.\\
    \[
        \begin{split}
        P(A=1|+) &= 0.5
        \\
        P(B=1|+) &= 0.5
        \\
        P(C=1|+) &= 1.0
        \\
        P(A=1|-) &= 0.333
        \\
        P(B=1|-) &= 0.333
        \\
        P(C=1|-) &= 0.333
        \end{split}
    \]

    2.\\
    \[
        \begin{split}
        P(+|A=1, B=1, C=1) &= \frac{P(+) * P(A=1|+)*P(B=1|+)*P(C=1|+)}{P(A=1,B=1,C=1)}
        \\\\
        &= \frac{P(+) * P(A=1|+)*P(B=1|+)*P(C=1|+)}{P(A=1,B=1,C=1)}
        \\\\
        &= \frac{0.4 * 0.5 * 0.5 *1}{P(A=1,B=1,C=1)}
        \\\\
        &= \mathbf{\frac{0.1}{P(A=1,B=1,C=1)}}
        \\\\
        P(-|A=1, B=1, C=1) &= \frac{P(-) * P(A=1|-)*P(B=1|-)*P(C=1|-)}{P(A=1,B=1,C=1)}
        \\\\
        &= \frac{P(-) * P(A=1|-)*P(B=1|-)*P(C=1|-)}{P(A=1,B=1,C=1)}
        \\\\
        &= \frac{0.6 * 0.333 * 0.333 *0.333}{P(A=1,B=1,C=1)}
        \\\\
        &= \frac{0.0222}{P(A=1,B=1,C=1)}
        \\
        \end{split}
    \]



\end{homeworkProblem}

\pagebreak
\begin{homeworkProblem}
    On the 28th page of the lecture notes “Lecture 3”, recalculate the likelihoods using m-estimate. Compare the m-estimate method and the original method
    shown on the 25th page for estimating probabilities. Which method is better and why?\\


    \textbf{Solution}\\
    \[
        \begin{split}
        P(HomO=Yes|No) &= \frac{3 + 3*2/3}{7+3}
        \\&=0.5
        \\
        P(HomO=No|No) &= \frac{4 + 3*2/3}{7+3}
        \\&=0.6
        \\
        P(HomO=Yes|Yes) &= \frac{0 + 3*1/3}{3+3}
        \\&=1/6
        \\
        P(HomO=No|Yes) &= \frac{3 + 3*1/3}{3+3}
        \\&=2/3
        \\
        P(Marital Status=Single|No) &= \frac{2+3*2/3}{7+3}
        \\&=0.4
        \\
        P(Marital Status=Divorced|No) &= \frac{1+3*2/3}{7+3}
        \\&=0.3
        \\
        P(Marital Status=Married|No) &= \frac{4+3*2/3}{7+3}
        \\&=0.6
        \\
        P(Marital Status=Single|Yes) &= \frac{2+3*1/3}{3+3}
        \\&=0.5
        \\
        P(Marital Status=Divorced|Yes) &= \frac{1+3*1/3}{3+3}
        \\&=1/3
        \\
        P(Marital Status=Married|Yes) &= \frac{0+3*1/3}{3+3}
        \\&=1/6
        \end{split}
    \]

    This is not complete, this needs to be normalized.
    \pagebreak
    \[
    \begin{split}
        P(HomO=Yes|No) &= 0.5/(0.5+0.6)
        \\&=0.455
        \\
        P(HomO=No|No) &= 0.6/(0.5+0.6)
        \\&=0.545
        \\
        P(HomO=Yes|Yes) &= (1/6)/(1/6+2/3)
        \\&=1/5
        \\
        P(HomO=No|Yes) &= (2/3)/(1/6+2/3)
        \\&=4/5
        \\
        P(Marital Status=Single|No) &= 0.4/(0.4+0.3+0.6)
        \\&=0.308
        \\
        P(Marital Status=Divorced|No) &= 0.3/(0.4+0.3+0.6)
        \\&=0.231
        \\
        P(Marital Status=Married|No) &= 0.6/(0.4+0.3+0.6)
        \\&=0.461
        \\
        P(Marital Status=Single|Yes) &= 0.5/(0.5+1/3+1/6)
        \\&=0.5
        \\
        P(Marital Status=Divorced|Yes) &= (1/3)/(0.5+1/3+1/6)
        \\&=1/3
        \\
        P(Marital Status=Married|Yes) &= (1/6)/(0.5+1/3+1/6)
        \\&=1/6
    \end{split}
    \]



\end{homeworkProblem}

\pagebreak

\begin{homeworkProblem}
    \textbf{Bayesian Belief Networks (Week 4)}\\
    If the person has high blood pressure, but exercises regularly and eats a healthy diet, 
    diagnose whether he has heart disease.
    \\
    \textbf{Solution}\\
    \[
    \begin{split}
        P(HD=Yes|BP=High, D=Healthy, E=Yes) &= \frac{P(HD=Yes, BP=High, D=Healthy, E=Yes)}{P(BP=High, D=Healthy, E=Yes)}
        \\\\
    \end{split}
    \]
    \[
    \begin{split}
        = \frac{P(BP=High|HD=Yes,D=Healthy,E=Yes)P(HD=Yes,D=Healthy,E=Yes)}{P(BP=High, D=Healthy, E=Yes)}&
        \\\\
        = \frac{P(BP=High|HD=Yes)P(HD=Yes,D=Healthy,E=Yes)}{P(BP=High, D=Healthy, E=Yes)}&
        \\\\
        = \frac{0.85 * P(HD=Yes|D=Healthy,E=Yes) * P(D=Healthy,E=Yes)}{P(BP=High| D=Healthy, E=Yes)*P(D=Heahtly,E=Yes)}&
        \\\\
        = \frac{0.85 * 0.25}{P(BP=High| D=Healthy, E=Yes)}&
    \end{split}
    \]
    \\\\\\
    \[
    \begin{split}
        P(HD=No|BP=High, D=Healthy, E=Yes) &= \frac{P(HD=No, BP=High, D=Healthy, E=Yes)}{P(BP=High, D=Healthy, E=Yes)}
        \\\\
    \end{split}
    \]
    \[
    \begin{split}
        = \frac{P(BP=High|HD=No,D=Healthy,E=Yes)P(HD=No,D=Healthy,E=Yes)}{P(BP=High, D=Healthy, E=Yes)}&
        \\\\
        = \frac{P(BP=High|HD=No)P(HD=No,D=Healthy,E=Yes)}{P(BP=High, D=Healthy, E=Yes)}&
        \\\\
        = \frac{0.2 * P(HD=No|D=Healthy,E=Yes) * P(D=Healthy,E=Yes)}{P(BP=High| D=Healthy, E=Yes)*P(D=Heahtly,E=Yes)}&
        \\\\
        = \frac{0.2 * 0.75}{P(BP=High| D=Healthy, E=Yes)}&
    \end{split}
    \]
    \pagebreak

    \[
    \begin{split}
        \frac{0.85 * 0.25}{P(BP=High| D=Healthy, E=Yes)} + \frac{0.2 * 0.75}{P(BP=High| D=Healthy, E=Yes)} &= 1
        \\\\
        \frac{0.3625}{P(BP=High| D=Healthy, E=Yes)}&= 1
        \\\\
        P(BP=High| D=Healthy, E=Yes) &= 0.3625
    \end{split}
    \]
    Therefore,
    \[
    \begin{split}
        P(HD=Yes|BP=High, D=Healthy, E=Yes) &= \frac{0.85 * 0.25}{0.3625}
        \\\\
        &=\mathbf{0.5862}
        \\\\
        P(HD=No|BP=High, D=Healthy, E=Yes) &= \frac{0.2*0.75}{0.3625}
        \\\\
        &=\mathbf{0.4138}
    \end{split}
    \]    



\end{homeworkProblem}
    



\end{document}